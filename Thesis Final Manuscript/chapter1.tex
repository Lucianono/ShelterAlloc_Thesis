\chapter{The Problem and Its Background}

	Addressing the shelter issues in Bulacan is the focus of this research. This chapter highlights the background and significance of the study. Additionally, the key points and objectives of the study will be discussed along with the study's scope and limitations.

\section{Background of the Study}

	The Philippines experiences typhoons, earthquakes, and volcanic eruptions each year, leaving people vulnerable and displaced because the country is situated within the Pacific Ring of Fire and directly in the path of the typhoon belt in the Pacific Ocean. Bulacan itself has faced severe impacts from typhoons, including Typhoon Ondoy in 2009 which affected over 250, 000 families and claimed nearly 100 lives  \parencite{James2009}. For instance, typhoon Haiyan, also known as Super Typhoon Yolanda, devastated the nation in 2013 and displaced over 4 million people, exposing severe shortcomings in the Philippine shelter allocation systems \parencite{Iuchi2019}, highlighting the urgent need for efficient and responsive disaster management strategies. Despite ongoing efforts, the country continues to face significant challenges in shelter allocation, like overcrowding, resource distribution inefficiencies, and poor accessibility.
	
	Overcrowding remains a persistent issue, with many shelters operating beyond their intended capacity, leading to unsafe and unsanitary conditions. Accessibility problems further worsen the situation, as shelters are often located at distances difficult for affected individuals, particularly those in rural or remote areas, to reach due to poor infrastructure.	Additionally, the inefficient distribution of resources like food, water, and medical supplies results in shortages and uneven support across shelters. Compounding these issues such as disaster preparedness team do not use data-driven decision-making increasing the likelihood of making incorrect decisions highlights the data and planning gaps that hinder effective decision-making and timely shelter assignments. This emphasizes the need on developing innovative approaches to improve shelter allocation in disaster preparedness. 
	
	The concept of shelter allocation has been a critical aspect of disaster management throughout history, when communities would seek refuge in caves or fortified structures during emergencies. However, formalized shelter allocation strategies emerged in the 20th century with the rise of civil defense measures during World War II, where air raid shelters were established in urban areas. These air raid shelters were used to ensure the safety of the population of Europe because of the civil casualties and weakening of their social and military morale, as seen in London, Berlin, and Paris \parencite{Flebus1941,Shakibamaesh2015}. Post-war, shelter allocation evolved in response to natural disasters, and humanitarian organizations developed temporary shelters for displaced populations. Later, in 1970 in Bangladesh, the inadequacy of early efforts became evident during the Bhola cyclone, which led to numerous deaths. A study showed that people lacked access to information about the cyclone warning and that appropriate shelters were rare, resulting in a devastating 300,000 fatalities, with some estimates being even higher and an estimated 4.8 million people were affected by the cyclone due to the lack of effective shelter allocation. \parencite{Mari2020}
	
	Shelter allocation refers to the process of assigning displaced communities to available shelters during disasters \parencite{Yin2023}. While shelter location-allocation,  focuses on determining shelter would be constructed, and then be used for displaced population \parencite{Xiujuan2019}. These process is crucial to disaster prevention and mitigation, ensuring the safety and well-being of affected populations by providing secure, accessible, and adequate temporary shelters. Effective shelter allocation involves considerations such as shelter capacity, proximity to disaster sites, and the specific needs of vulnerable communities. 
	
	Over the years, strategies have evolved from reactive measures to a more systematic approach incorporating modern technologies and methodologies, highlighting the increasing importance of structured and efficient shelter allocation. However, these challenges, particularly in developing countries like the Philippines, where limited infrastructure hinders its effective implementation.
	
	This study proposes a data-driven solution ensuring an effective and feasible implementation of shelter allocation. This would be by developing a decision support system using a genetic algorithm (GA), a computational method inspired by natural selection and evolution, to address the inefficiencies in shelter allocation. GA optimize solutions by mimicking the process of evolution such as selection, crossover, and mutation, to identify best outcome. \textcite{Mathew2012} discussed that these algorithms represent potential solutions to a given problem using a basic chromosome-like data structure and employ recombination operators to maintain essential information, making it ideal to solve wide variety of problems. In shelter allocation, GA can optimize the assignment process by simultaneously considering factors such as shelter capacity, location, and accessibility. This approach aims to minimize overcrowding, improve resource distribution, and enhance overall efficiency in managing shelters during disasters.
		
	Given the Philippines' frequent exposure to natural disasters, implementing a shelter allocation system. Calumpit, Bulacan was chosen for its high vulnerability to flooding since it has been a catch basin of floodwaters from their neighboring areas. Despite the multitude of studies formulating models to address the shelter location-allocation problem, there remains a lack of system integration. This gap makes it difficult for decision-makers, particularly those unfamiliar with the mathematical models proposed in various studies, to apply them effectively in their decisions. This study is dedicated to developing a programmed system that uses genetic algorithm to address the challenges experienced in the Philippines, focusing on Calumpit, Bulacan. The proposed shelter location-allocation system has the potential to significantly improve disaster response efficiency, and enhance safety for displaced individuals.The insights gained from this study could be applied to other municipalities facing similar challenges, contributing to the national effort to strengthen disaster resilience. 
	%\nocite{*}.

\section{Statement of the Problem}
	This study aims to address the shelter allocation problem for the municipality of Calumpit, particularly during disaster situations such as storm surges and flash floods. The primary challenge is how to develop a decision support system integrating a model that solves location-allocation problem to ensure that victims are allocated to shelters in an optimal manner. Specifically, the paper aims to solve the following questions:
	
	\begin{enumerate}
		\item What is the optimal shelter location-allocation plan for the municipality of Calumpit?
		\item How can the proposed system be developed featuring the following functionalities:
		\begin{enumerate}
			\item Data Modification,
			\item Model Modification,
			\item Data Simulation,
			\item Shelter Tagging, and
			\item Report Protection?
		\end{enumerate}
		\item How acceptable is the proposed system based on the criteria defined in Technology Acceptance Model?
		\begin{enumerate}
			\item Perceived usefulness,
			\item Perceived ease of use,
			\item Attitude towards using, and
			\item Behavioral intention?
		\end{enumerate}
		\item How well does the proposed system meet the ISO / IEC 25010 requirements?
		\begin{enumerate}
			\item Functional Suitability,
			\item Performance Efficiency,
			\item Compatibility,
			\item Interaction Capability,
			\item Reliability,
			\item Security,
			\item Maintainability, and
			\item Flexibility?
		\end{enumerate}
	\end{enumerate}
	
\section{Objective of the Study}
	The primary objective of this study is to develop an acceptable Decision Support System (DSS) for optimal shelter location-allocation in disaster situations, specifically designed for the municipality of Calumpit. To be able to achieve, the study is guided by the following specific objectives:
	
	\begin{enumerate}
		\item \textbf{Model Adoption}: Use of Bilevel No Transfer model, which minimizes travel distance and shelter construction/maintenance costs. The model is known to have a feasible solution for optimal shelter allocation.

		\item \textbf{Data Collection}: Data will be gathered from Local Government Units(LGUs) of Calumpit, and from online databases such as from Philippine Statistics Authority(PSA). This includes barangay locations, populations, vulnerable groups, candidate shelter locations, shelter capacities, and maintenance/construction costs. Local administrators will assist in providing accurate and up-to-date data for the study.
		
		\item \textbf{System Development}: Develop a DSS integrating the adopted model, and solving the model using Genetic Algorithm. The system should produce an output containing the optimal shelter location-allocation of the municipality. The user of the system should also have access to modify the data, tag the shelters, and secure the reports.
		
		\item \textbf{System Assessment}: The acceptability of the proposed system will be determined by a survey based on the criteria of Technology Acceptance Model(TAM) and ISO/IEC 25010. This will be given to the potential users of the system such as the Municipal Disaster Risk Reduction Management Office of Calumpit.
		
	\end{enumerate}

\section{Significance of the Study}
	The research seeks to improve disaster preparedness by providing a program and a model that facilitates more efficient decision-making and optimizes shelter placements regarding accessibility, cost-efficiency, and community proximity.The proposed solution has the potential to significantly reduce the risks associated with natural disasters in that area and enhance the overall management of shelter allocation and emergency resources.
	
	The resulting shelter location-allocation from this project will primarily benefit the citizens of disaster-prone areas in Calumpit. This leads to faster evacuation times for use in the welfare of its citizens. The resulting optimization of shelter locations also extends to the efficient use of financial and logistical resources, minimizing the costs required in the maintenance and operation of shelters, thus allowing reallocation of previously spent resources into other areas.
	
	By providing insights into optimized shelter locations and evacuee distribution, this project seeks to benefit the following parties:
	
	\begin{description}
		\item[Local Government.] The local government of the target municipality is a beneficiary of this project due to being responsible for disaster management and response within their jurisdiction. The administration of the affected municipalities’ LGU would achieve a system that will assist in the evacuation and protection of citizens and thus may divert their attention elsewhere into other areas.
		
		\item[Communities.] The local communities within Calumpit that are directly affected by disasters, especially floods, are important beneficiaries of this project. The community would achieve faster response times, shelters that are located in optimal positions that take into account their homes and workplaces, and more systematic procedures in evacuation.
		
		\item[Future researchers.] Future researchers are a beneficiary due to this project being open to the public and thus, researchers and developers may derive their own thesis projects that are similar for use in different areas of the Philippines.
		
	\end{description}

\section{Scope and Limitations}
	This thesis will be limited only on identifying and assigning evacuation shelters, referred to as shelter location-allocation. This is geographically limited on Calumpit due to being the most affected municipality in Bulacan on typhoons. However, this study can be applied to other disaster-prone areas with some adjustments, which will not be covered in this thesis. Moreover, the system will use Bilevel No Transfer model through Genetic Algorithm only.
	
	The thesis will utilize real-world information from Calumpit, including existing shelter locations and geographic layouts, as well as cost estimates for shelter maintenance and other operations. This data will be acquired by the researchers with the help of the local government units of Calumpit.
	
	In technical aspect, the proposed system is limited to the Windows operating system (OS) due to its accessibility and familiarity among the general population. Additionally, Windows provides a stable environment and robust support for applications, further justifying its selection as the target OS for the proposed system.
	
	The project is limited to a 10-month period wherein all stages of the thesis and consequent testing must be completed. Updates on data, and project maintenance beyond this period will not be covered by the researchers.
