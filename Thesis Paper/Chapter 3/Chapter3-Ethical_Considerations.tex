\section{Ethical Considerations}

Data gathered contained sensitive data so that ethical considerations should be practiced throughout the research.  Informed consent will be obtained from each participant, providing them with a clear understanding of the research purpose, procedures, and their rights to withdraw at any time. To protect the participants’ privacy, all responses will be anonymized and handled with strict confidentiality. Additionally, data usage will be restricted to research purposes only, ensuring that participants' information remains safeguarded throughout the research process and beyond. 

\textbf{Informed Consent:} All respondents are informed about the purpose, procedures, and potential impacts of the study before they agree to participate. The respondents will be given a detailed explanation of the study, including the nature of their involvement, the voluntary nature of participation, and their right to withdraw at any time.
\textbf{Confidentiality and Anonymity:} Personal identifiers are removed from the data to ensure that the respondents cannot be traced back to other respondents. Data are stored securely, and access will be restricted to the researchers. Any publications or presentations resulting from the study will use aggregated data with no individual responses disclosed, safeguarding the identity and privacy of all respondents.
\textbf{Data Protection:} The study will adhere to relevant data protection regulations, such as local data protection laws, ensuring the secure handling and storage of all collected data. Informing the respondents about how their data will be collected, stored, and used. Their rights to privacy and data protection will be respected, and only authorized researchers will have access to the information.
\textbf{Minimizing Harm:} The study will be structured to minimize any potential harm to respondents. This includes ensuring that all questions are respectful, non-discriminatory, and that the data collection process does not inconvenience the respondents. Any potential risks will be clearly communicated, and measures will be implemented to mitigate these risks and uphold the respondents well-being throughout the study.
\textbf{Transparency and Honesty: } The researchers will uphold transparency and honesty throughout every stage of the study. This commitment includes accurately reporting research findings, acknowledging limitations, and strictly avoiding data manipulation or bias. Respondents will be informed of the study's progress, purposes, and outcomes, ensuring they remain fully aware of how their contributions are utilized and valued within the research.
By adhering to these ethical principles, the study will safeguard respondents' rights and well-being, uphold the integrity of the research process, and enhance the credibility and reliability of its findings. This ethical approach shows trust between researchers and respondents, contributing to meaningful and responsible research outcomes.
