\section{Research Design}

The research design for this study is mixed-method applied research, this approach combines both quantitative and analytical methods to provide a structured framework for development and evaluation of the shelter location allocation system.

\subsection{Applied Research}

The applied research aspect of this study focuses on developing and implementing a practical solution for shelter location-allocation in real-world scenarios. The primary objective is to create a shelter location-allocation system using a genetic algorithm.
This applied research approach aims to deliver a functional and reliable shelter location-allocation system that local government units (LGUs) and disaster response agencies can easily deploy and adapt. By emphasizing incremental progress and user-centered development, this ensures that the research outputs are directly applicable and beneficial to communities in disaster-prone areas.

\subsection{Quantitative Research}

The quantitative research component of this study aims to measure and analyze the acceptability of the shelter location allocation using the ISO/IEC 25010 standard. The approach will provide objective and measurable evidence to evaluate the system's usability, reliability, and overall quality performance. The quantitative methods will seek to quantify the use of genetic algorithms in the shelter location allocation system, providing valuable insights into the system's effectiveness for emergency management.
Methods will include collecting user feedback and performance metrics through structured surveys and system testing to assess these attributes. The structured survey will gauge user satisfaction with system usability and functionality using a 10-point Likert scale, providing quantifiable insights into the system's ease of use and accessibility. Additionally, response times and the accuracy of shelter allocations generated by the genetic algorithm will test the system's reliability, further collecting quantitative performance metrics during systems testing and the structured survey.
Incorporating quantitative research in this study is justified as it can provide empirical evidence supporting the shelter allocation system's acceptability and effectiveness.

\subsection{Qualitative Research}

The researchers will use qualitative research to understand the current issues of shelter location allocation from the stakeholders' perspectives. This approach requires engaging with the PDRRMC, PSWDC, MDRRMC, and MSWDC offices to identify the additional requirements for the system. This ensures that the development of the shelter location allocation system will address real-world challenges and support those involved in evacuating communities during natural disasters and be of acceptable quality.
The researchers chose a questionnaire survey to achieve this. A questionnaire survey was chosen to provide a structured way to gather detailed feedback from the stakeholders about their experiences, challenges, information about the shelters, and expectations regarding shelter location allocation and this study. This method allows the researchers to collect various opinions and insights from various stakeholders, which is important for understanding the complex issues involved in disaster management and shelter allocation.
This insight is important for ensuring the system's effectiveness, as it helps identify the challenges to ensure the proper functioning of the shelter location allocation system and meet the needs of the affected communities. By capturing these detailed perspectives, the system will be more practical, user-friendly, and responsive to the real-world needs of disaster management.
