\section{System Evaluation}

The evaluation instrument is a structured questionnaire designed to assess the acceptability of the shelter location allocation system based on the key quality attributes outlined in the ISO/IEC 25010 standard. The questionnaire will evaluate various aspects of the system, including usability, reliability, performance efficiency, and overall user satisfaction. Through this instrument, the study aims to gather objective data that will help determine how well the system meets user needs and expectations.

\textbf{Functional Suitability:} This evaluates how well the system meets user needs and performs its intended functions. The questions will focus on the system's ability to allocate shelters effectively, manage community and shelter information, and provide accurate data.
\textbf{Performance Efficiency:} This attribute will evaluate the system's response time and processing capacity. The questionnaire will include items assessing the system's speed in performing functions, efficient use of resources such as memory and processing power, and ability to handle maximum load requirements. By examining performance efficiency, the study aims to determine whether the system can perform reliably under diverse operational conditions.
\textbf{Compatibility:} This will assess the system’s ability to exchange and use information with other systems. The questionnaire will ask how well the system integrates with external databases, applications, or systems. 
\textbf{Usability:} This will prioritize the system's usability, learnability, and overall user satisfaction. The questionnaire will gather detailed user feedback regarding their experiences with the system. The questionnaire will address aspects such as ease of navigation, effectiveness in completing tasks, and general user engagement.
\textbf{Reliability:} This attribute will assess the system's interaction capability, focusing on its ease of use, appropriateness for users, and ability to guide users in completing tasks effectively. The evaluation will include questions on how easily users can recognize if the system meets their needs, learn its functions, and operate it with minimal errors. Additional focus will be on user engagement, inclusivity, and the availability of assistance to support a diverse range of users. 
\textbf{Security:} The system ensures that all collected data will remain confidential and solely for this research. No data will be uploaded online, and all information will be stored offline. Access to data will be limited to the researchers and users.
\textbf{Maintainability:} This will evaluate the system's ease of adaptability to meet new requirements, correct errors, and improve performance. The questionnaire will assess the system's reusability, analyzing how parts of the system can be leveraged in future developments or other systems, and its testability, ensuring that any necessary changes can be efficiently implemented and quickly identified so that testing can be done efficiently. 
\subsection{Population and Sample}
The population of this study will consist of PDRRMC, PSWDC, MDRRMC, and MSWDC of the municipality of Marilao. The diversity of the perspective of these are important for the comprehensive evaluation of the system’s acceptability.
\textbf{PDRRMC:} The staff consists of Provincial Disaster Risk Reduction Management Council members of Bulacan, who are responsible for formulating and implementing the province's disaster risk reduction plan. Their input is important in aligning the shelter location allocation system.
\textbf{PSWDC:} The Provincial Social Welfare and Development Council in Bulacan plays a key role in managing the evacuation and welfare of the communities during disasters. Their expertise is essential in ensuring that the shelter allocation system aligns with evacuees' specific needs, such as accessibility and capacity to the shelter location allocation across the province.

\textbf{MDRRMC:} This group comprises individuals within the MDRRMC who are responsible for disaster preparedness, response, and recovery operations at the municipality of Marilao. Their insights are vital for understanding the operational requirements considerations for implementing the shelter location allocation system, ensuring it aligns with established disaster response protocols and local needs.
\textbf{MSWDC:}  This group includes personnel from the MSWDC who are responsible for managing evacuees during disaster events. Their perspectives are crucial for understanding the needs and challenges involved in shelter allocation, particularly in ensuring that evacuee management is effectively integrated into the system’s functionality. Their input will help tailor the system to address the welfare requirements of evacuee management during emergencies.

Since the population is small, the study will include the entire population as the sample to ensure comprehensive representation, to capture the perspectives of different stakeholder groups proportionately.

\subsection{Data Collection Procedures}
Data collection for this project will be conducted in two main phases: the first focused on gathering data for system development, and the second on system evaluation.

\textbf{For System Development:} This phase aims to establish a foundational understanding of current disaster-response processes, along with gathering essential data for the system. Interviews will be conducted with representatives from local government units (LGUs) specifically from the target population to understand their existing shelter allocation process, challenges, and expectations for a decision support system. 

Additionally, to ensure that the system produces outputs, data on shelters, disaster events, and typhoons will be collected. This includes accessing records on past typhoon impacts, available shelter locations, and their respective capacities. Community or barangay data will be sourced from publicly available databases, such as PhilAtlas, which provides detailed geographical and demographic information essential for accurate system modeling.

\textbf{For System Evaluation:} After system development, an evaluation phase will gauge the system’s acceptability and user satisfaction. To do this, a questionnaire will be developed based on the ISO/IEC 25010 software quality model, which will employ a 10-point Likert scale, allowing participants to provide precise feedback on each quality criterion. The system will be demonstrated to a selected sample population in a face-to-face presentation, where they will have the chance to interact with the system firsthand. This will be followed by the completion of the questionnaire, enabling participants to provide feedback based on their experience using the system.

Through this structured, two-phase data collection process, the study aims to gather comprehensive and reliable information to support the development and evaluation of the shelter location allocation system. Following ISO/IEC 25010 standards ensures that the system meets its intended objectives and offers insights into potential areas for improvement, which will be critical for broader implementation in disaster-response operations. Through the data collection process, the researchers will ensure that all ethical considerations are thoroughly addressed to uphold the integrity and privacy of the participants

\subsection{Data Processing and Analysis}
Once data collection is complete, the gathered information will undergo several stages: cleaning, processing, analysis, and interpretation. This ensures it meets the needs of the system and allows for an accurate assessment of its effectiveness.

All collected data will first be cleaned to remove any inconsistencies, duplicates, or irrelevant information. This process is crucial for ensuring the data’s quality and reliability, particularly for a system that relies on precision in both inputs and outputs. Any insufficient data will be asked on the LGU to fill in the gaps of the datasets. 

For justification of the target area’s vulnerability, typhoon data spanning from 2006 to 2022 will be examined. This dataset includes the number of affected and evacuated individuals across various municipalities in Bulacan. The data will be grouped by municipality, summing the affected and evacuated residents over the specified period. This analysis helps justify the focus on Bulacan and supports the rationale for the shelter allocation model’s implementation in the municipality.

The cleaned shelter and community data will be fed into the system to facilitate accurate shelter location allocation modeling. The system will use these inputs to generate optimized shelter placement recommendations based on the model adopted.

The structured survey, which is the second phase of data collection, will be given to the PDRRMC, PSWDC, MDRRMC, and MSWDC staff to evaluate the system's acceptability and provide critical feedback for further refinement.

This study's data processing will involve structured analysis to interpret findings from the acceptability survey, aligning with the ISO/IEC 25010 standard. This standard evaluates key quality attributes, and each attribute will be assessed using a 10-point Likert scale. 1 being the lowest or extremely disagree, and 10 being the highest or extremely agree. The Table below shows the description of each numerical data.

\begin{table}[]
	\begin{tabular}{ll}
		1  & Extremely Disagree   \\
		2  & Moderately Disagree  \\
		3  & Little more disagree \\
		4  & Mildly disagree      \\
		5  & Partially disagree   \\
		6  & Partially agree      \\
		7  & Mildly agree         \\
		8  & Little more agree    \\
		9  & Moderately agree     \\
		10 & Extremely agree     
	\end{tabular}
\end{table}

After the data collection and summarization, the 10-point Likert scale will categorize the acceptability levels. This dual use of data allows both the system functionality and its user acceptability to be comprehensively assessed. 
