\section{Background of the Study}

The Philippines experiences typhoons, earthquakes, and volcanic eruptions each year, leaving people vulnerable and displaced because the country is situated within the Pacific Ring of Fire and directly in the path of the typhoon belt in the Pacific Ocean. Typhoon Haiyan, also known as Super Typhoon Yolanda, devastated the nation in 2013 and displaced over 4 million people, exposing severe shortcomings in the Philippine shelter allocation systems \parencite{Iuchi2019}. Despite ongoing efforts, the country continues to face significant challenges in shelter allocation. Overcrowding remains a persistent issue, with many shelters operating beyond their intended capacity, leading to unsafe and unsanitary conditions. Accessibility problems further worsen the situation, as shelters are often located at distances difficult for affected individuals, particularly those in rural or remote areas, to reach due to poor infrastructure.


Additionally, the inefficient distribution of resources like food, water, and medical supplies results in shortages and uneven support across shelters. The needs of vulnerable groups, such as the elderly, disabled, and those with chronic health conditions, are frequently overlooked, increasing their risk during disasters. Compounding these issues are data and communication gaps that hinder effective decision-making and timely shelter assignments, emphasizing the need for a more efficient and responsive shelter allocation system. The recurring nature of these disasters highlights the urgent need for innovative approaches to develop a robust system capable of meeting the high demand for effective shelter allocation.

Historically, it has been a critical aspect of disaster management since ancient times when communities would seek refuge in caves or fortified structures during emergencies. However, formalized shelter allocation strategies emerged in the 20th century with the rise of civil defense measures during World War II, where air raid shelters were established in urban areas. These air raid shelters were used to ensure the safety of the population of Europe because of the civil casualties and weakening of their social and military morale, as seen in London, Berlin, and Paris \parencite{Flebus1941,Shakibamaesh2015}. Post-war, shelter allocation evolved in response to natural disasters, and humanitarian organizations developed temporary shelters for displaced populations. Later, in 1970 in Bangladesh, the inadequacy of early efforts became evident during the Bhola cyclone, which led to numerous deaths. As cited by \textcite{Hossain2018}, a study showed that people lacked access to information about the cyclone warning and that appropriate shelters were rare, resulting in a devastating 300,000 fatalities, with some estimates being even higher and an estimated 4.8 million people were affected by the cyclone due to the lack of effective shelter allocation.

Shelter location-allocation (or shelter allocation) refers to assigning displaced communities to available shelters during a disaster \parencite{Yin2023}. It is a crucial aspect of disaster prevention and mitigation that involves the safety and well-being of affected populations by providing secure, accessible, and adequate temporary shelter. Effective shelter allocation involves considerations such as shelter capacity, proximity to disaster sites, and the specific needs of vulnerable communities. Over the years, strategies have evolved from reactive measures to a more systematic approach incorporating modern technologies and methodologies, highlighting the increasing importance of structured and efficient shelter allocation. However, ensuring sufficient and accessible shelter persists, especially in developing countries like the Philippines.

This study proposes using a genetic algorithm (GA), a computational method inspired by natural selection and evolution, to address the inefficiencies in shelter allocation. \textcite{Mathew2012} discussed that these algorithms represent potential solutions to a given problem using a basic chromosome-like data structure and employ recombination operators to maintain essential information. Genetic algorithms are commonly regarded as tools for function optimization, mimicking the process of evolution by generating solutions to optimization problems through selection, crossover, and mutation. This algorithm has been applied to a wide variety of problems. In shelter allocation, genetic algorithms can optimize the assignment process by simultaneously considering factors such as shelter capacity, location, and accessibility. This approach aims to minimize overcrowding, improve resource distribution, and enhance overall efficiency in managing shelters during disasters.

Given the Philippines' frequent exposure to natural disasters, implementing a shelter allocation system is needed. This study is dedicated to developing a system based on genetic algorithms to address the unique challenges experienced in the Philippines, focusing on Hagonoy, Bulacan, chosen for its high vulnerability to flooding and its challenges in managing shelter allocation during disasters. The proposed shelter allocation system has the potential to significantly improve disaster response efficiency, reduce overcrowding, and enhance safety for displaced individuals. Implementing a shelter allocation system using genetic algorithms in Hagonoy, Bulacan, is expected to improve disaster response significantly. By optimizing shelter allocation, the system can reduce overcrowding, improve access to resources, and enhance overall safety for displaced individuals. Additionally, the insights gained from this study could be applied to other municipalities facing similar challenges, contributing to the national effort to strengthen disaster resilience.


