\section{Other Allocation Framework}

The following articles and studies are chosen for their use of similar allocation methods in systems, however, these uses vary and were not applied to shelter evacuation or other similar topics.  These articles may have uses not quickly noticed and may give inspiration for the allocation methods used in this thesis.

\begin{longtable}{|L{.2\linewidth}|L{.74\linewidth}|}
	\hline
	\multicolumn{2}{|L{.94\linewidth}|}{\fullcite{Yahyaei2019}}\\ \hline
	\textbf{Summary.} & This paper addresses the design of a relief logistics network under uncertainty and the risk of facility disruption, applying robust optimization to handle uncertainties and conducts computational experiments to perform sensitivity analysis. The study also finds that as disruptions increases, the network prioritizes establishing more reliable facilities. \\ \hline
	\textbf{Critique.} & The paper did not make use of the Genetic Algorithm nor the Hierarchical Location Allocation Model toward the allocation of shelters. \\ \hline
	\textbf{Similarities and Differences.} & The paper is quite similar in function towards the thesis, the paper may provide insight for alternative methods not immediately realized in the thesis	The paper compared to the thesis takes into account supply facilities in their shelter allocation whilst the thesis does not.\\ \hline
\end{longtable}

\begin{longtable}{|L{.2\linewidth}|L{.74\linewidth}|}
	\hline
	\multicolumn{2}{|L{.94\linewidth}|}{\fullcite{Donghai2020}}\\ \hline
	\textbf{Summary.} & This paper proposes a mixed integer linear programming model for addressing the post-disaster resettlement problem, focusing on both waiting cost (human suffering) and fairness. The model incorporates two fairness indicators to ensure equity throughout the entire resettlement process and during specific time periods. It is implemented in GAMS and solved using the CPLEX solver. An illustrative example is presented to clarify the model's features, and a case study on the Yushu earthquake is conducted to demonstrate its practical application in real-world scenarios.\\ \hline
	\textbf{Critique.} & The paper does not have a clear explanation of the significance of the integration of waiting costs with fairness considerations. It also lacks detail on the results of the illustrative example and the Yushu earthquake case study.\\ \hline
	\textbf{Similarities and Differences.} & The paper gives the researchers another model to work with for shelter allocation this time giving consideration to waiting costs.	The paper makes no use of Genetic Algorithm nor the HIER model.\\ \hline
\end{longtable}

\begin{longtable}{|L{.2\linewidth}|L{.74\linewidth}|}
	\hline
	\multicolumn{2}{|L{.94\linewidth}|}{\fullcite{Shaoqing2021}}\\ \hline
	\textbf{Summary.} & The article proposes a hybrid approach combining fuzzy-VIKOR and bi-objective programming to optimize emergency shelter location-allocation decisions. The method considers multiple criteria such as topography, geological type, slope, vegetation, and power facilities as well as quantitative factors like distance and capacity; Used in evaluating and ranking the candidate shelters while the bi-objective programming model optimizes the location-allocation to minimize the distance and maximize suitability.\\ \hline
	\textbf{Critique.} & The article assumes the preferences and weights for the criteria are known and fixed, which may not be the case in real-world scenarios. The article does not consider the dynamic nature of emergency shelter location-allocation problems, where the demand and supply may change over time.\\ \hline
	\textbf{Similarities and Differences.} & The relevance lies on the methods and common goals that are used to search for optimal solutions that balance multiple criteria, such as distance, capability and suitability	The proposed hybrid approach differs in the uses of bi-ovjective programming and fuzzy-VIKOR to evaluate and optimize shelter location, whereas genetic algorithm use fitness function to evolve a population of candidate solutions, resulting in distinct solution representations and evaluation functions.\\ \hline
\end{longtable}

\begin{longtable}{|L{.2\linewidth}|L{.74\linewidth}|}
	\hline
	\multicolumn{2}{|L{.94\linewidth}|}{\fullcite{Bo2021}}\\ \hline
	\textbf{Summary.} & The paper proposes a mixed-integer programming MIP model to optimize the location and resources distribution of emergency warehouses under certainty. The authors introduced uncertain scenarios to simulate different disaster degrees. To solve the MIP model, Particle Swarm Optimization PSO and Variable Neighborhood Search VNS are designed and compared with a commercial solver CPLEX.\\ \hline
	\textbf{Critique.} & The paper assumed that the demand for resources in disaster areas is known.\\ \hline
	\textbf{Similarities and Differences.} & The paper focuses on optimizing the location under uncertainty. The use of PSO and VNS to solve the MIP model can be seen as similar to the approach used in GA to solve shelter location allocation problems as both PSO and VNS are population based that uses iterative search processes to find the optimal solution, while GA use evolutionary principles to search for the optimal solution.	The problem formulation of the paper differs because of the use of a mixed-integer programming model, whereas shelter location allocation problems are formulated as a multi-objective optimization problem. The use of uncertain scenarios to simulate different disasters is not considered in shelter location allocation problems.\\ \hline
\end{longtable}

\begin{longtable}{|L{.2\linewidth}|L{.74\linewidth}|}
	\hline
	\multicolumn{2}{|L{.94\linewidth}|}{\fullcite{Anak2023}}\\ \hline
	\textbf{Summary.} & This paper proposed a stochastic model for multi-objective location-routing for creating a humanitarian network for pre-disaster response. The model aims to minimize the overall costs of the network’s setup, the time required to travel through it, and the number of vehicles necessary for transferring affected individuals to evacuation centers. The model were solved by using MOPSO and MOSA. \\ \hline
	\textbf{Critique.} & They did get primary and secondary data from PDRRMO, but they dod not present it on the paper. They also stated that their model only considered one type of vehicle. \\ \hline
	\textbf{Similarities and Differences.} & The target location which is the Philippines is same as us. This may help us consider other factors for disasters' assesssment specifically experienced locally.	The paper focuses on warehouse location and resources distribution. They also did a multi-objective approach to solve the model. \\ \hline
\end{longtable}

\begin{longtable}{|L{.2\linewidth}|L{.74\linewidth}|}
	\hline
	\multicolumn{2}{|L{.94\linewidth}|}{\fullcite{Mumtaz2021}}\\ \hline
	\textbf{Summary.} & The paper proposes a methodology for locating and allocating search and rescue SAR assets, which include boats and helicopters to enhance the performance of maritime SAR missions. The authors develop a dynamic multi-objective mixed integer linear programming model that incorporates simulated incident scenarios. \\ \hline
	\textbf{Critique.} & The study focused on the Aegean Sea region and the model’s complexity may make it challenging to apply to other areas. Additionally, the study relies on historical data. \\ \hline
	\textbf{Similarities and Differences.} & The approach to optimizing SAR resource allocation can be relevant  to shelter location allocation with genetic algorithms because it involves optimizing multi-objective and accounts for uncertainty and variability through simulation.	The difference lies in the specific problem context and the optimization approached used since the study focuses on SAR assets and uses a mixed-integer linear programming MILP approach. \\ \hline
\end{longtable}
