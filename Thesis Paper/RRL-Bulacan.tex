\section{Bulacan}

The next few articles have been used as proof of the relevance of the thesis in that the province of Bulacan is very vulnerable and greatly affected by floods and other natural disasters.

\begin{longtable}{|L{.2\linewidth}|L{.74\linewidth}|}
	\hline
	\multicolumn{2}{|L{.94\linewidth}|}{\fullcite{GMA2019}}\\ \hline
	\textbf{Summary.} & This news article talked about the flood along MacArthur Highway in Marilao, Bulacan due to a weekend of raining. As of 7 am of that day the flood remains gutter deep.\\ \hline
	\textbf{Critique.} & The article is from a news source, which - while from a reputable group - may not be as reliable as an article from a journal or study.\\ \hline
	\textbf{Similarities and Differences.} & The news article shows the problem and danger of floods in Marilao, Bulacan being able to stay for long periods of time. The article covers Marilao whilst the thesis covers Hagonoy. However, this article may still be taken into account due to both Hagonoy and Marilao having similar states of protection against floods. \\ \hline
\end{longtable}

\begin{longtable}{|L{.2\linewidth}|L{.74\linewidth}|}
	\hline
	\multicolumn{2}{|L{.94\linewidth}|}{\fullcite{Carmela2021}}\\ \hline
	\textbf{Summary.} & The article reports that heavy rainfall caused by the southwest monsoon and Typhoon Fabian has flooded 35 villages in five towns in Bulacan, including Hagonoy and Marilao. The flood has made roads impassable to light vehicles, and the Angat Dam’s water level has risen to 185.72 meters above sea level.\\ \hline
	\textbf{Critique.} & The news article highlights the importance of considering flood risk in shelter location allocation. However, it lacks comprehensive analysis of the underlying factors contributing to flooding in Hagonoy and Marilao, Bulacan. Additionally it is a news article and does not provide a detailed account of the shelter location allocation strategies.\\ \hline
	\textbf{Similarities and Differences.} & The article’s findings are highly relevant to shelter location allocation, considering Hagonoy and Marilao as potential locations. The genetic algorithm can take account of flood risk data to optimize shelter locations. The difference is that even though Hagonoy and Marilao lie in the flood risk data; both areas are prone to flooding, the severity and frequency of flooding may vary. The genetic algorithm should be tailored to each location, taking into account the specific flood risk data for each area.\\ \hline
\end{longtable}

\begin{longtable}{|L{.2\linewidth}|L{.74\linewidth}|}
	\hline
	\multicolumn{2}{|L{.94\linewidth}|}{\fullcite{Jacque2022}}\\ \hline
	\textbf{Summary.} & The newspaper article reports on the concerns of Hagonoy residents in the Philippines regarding frequent flooding in the area, urging the government to take action to address the issue which has been affecting their livelihoods; calling for the construction of a flood control system.\\ \hline
	\textbf{Critique.} & The newspaper article focuses on the concerns of the residents and lacks specific details on the causes of flooding and the proposed solutions. It only serves as a source of information about flooding in Hagonoy, Bulacan and not specifically a source of information about Shelter Location-Allocation.\\ \hline
	\textbf{Similarities and Differences.} & The article focuses on flooding and the need for effective solutions to mitigate its impact gives relevance to the concept of shelter location allocation.There is no difference as it does not directly relate to shelter location allocation and only justifies the location that will be used on the shelter location allocation.\\ \hline
\end{longtable}

\begin{longtable}{|L{.2\linewidth}|L{.74\linewidth}|}
	\hline
	\multicolumn{2}{|L{.94\linewidth}|}{\fullcite{Iya2023}}\\ \hline
	\textbf{Summary.} & The news article emphasizes how victims of flood lived through in Bulacan. It also exposed that province was placed under a state of calamity due to Typhoon Egay, affecting over 200,000 families and forcing many to evacuate.\\ \hline
	\textbf{Critique.} & The article focuses on human impact, but lacks on actual measurements or data. This is also a news whch may not be be as reliable as an article from a journal or study.\\ \hline
	\textbf{Similarities and Differences.} & It shows the importance of our system in the province of Bulacan due to the impacts and effects of the disaster The article is a from a news source and did not specify any allocation system used.\\ \hline
\end{longtable}