\section{Genetic Algorithm}

These articles, papers, and studies contain the use of Genetic Algorithm within several topics, these may provide insight for the thesis' own use of Genetic Algorithm.

\begin{longtable}{|L{.2\linewidth}|L{.74\linewidth}|}
	\hline
	\multicolumn{2}{|L{.94\linewidth}|}{\fullcite{Mathew2012}}\\ \hline
	\textbf{Summary.} & The paper provides an lecture on Genetic Algorithm and on its restrictions, uses, and niches. The paper also explains the processes and concepts of mutation, offspring, fitnesses, and crossovers\\ \hline
	\textbf{Critique.} & The paper is simply about Genetic Algorithm and does not contain any other relevancies toward other topics related to the thesis.\\ \hline
	\textbf{Similarities and Differences.} & The paper may be relevant for the researchers and for the thesis by providing insights of how Genetic Algorithm may be applied and offers guidance on how to adjust a Genetic Algorithm for shelter allocation 	The paper is an alternative explanation of Genetic Algorithm compared to the study originally used by the researchers.\\ \hline
\end{longtable}

\begin{longtable}{|L{.2\linewidth}|L{.74\linewidth}|}
	\hline
	\multicolumn{2}{|L{.94\linewidth}|}{\fullcite{Eyal2020}}\\ \hline
	\textbf{Summary.} & The book explains the concept of Genetic Algorithm and provided hands-on laboratory exercises regarding to the topic. It also exposes the different practices and application of the algorithm. \\ \hline
	\textbf{Critique.} & Explanation of the Genetic Algorithm is not mathematical based and has no proving involved. This is because this is more like a technical book.\\ \hline
	\textbf{Similarities and Differences.} & Justification and definition of the methods on our genetic algorithms are found on this book.	This is a lesson or lecture book. It is not a research. \\ \hline
\end{longtable}

\begin{longtable}{|L{.2\linewidth}|L{.74\linewidth}|}
	\hline
	\multicolumn{2}{|L{.94\linewidth}|}{\fullcite{Edgar2020}}\\ \hline
	\textbf{Summary.} & This paper explores the integration of semantics into Multi-Objective Genetic Programming (MOGP) by comparing three methods: Semantic Similarity-based Crossover (SSC), borrowed from single-objective GP, Semantic-based Distance as an additional criterion (SDO), and Pivot Similarity SDO. The study finds that optimizing semantic distance as an additional criterion improves performance over canonical methods and SSC. However, methods effective in single-objective GP don't always translate well to MOGP. The use of pivots, referencing sparse search regions, enhanced diversity and performance in the MOGP context.\\ \hline
	\textbf{Critique.} & The paper has an over-reliance on the assumption that semantics universally enhance MOGP performance.\\ \hline
	\textbf{Similarities and Differences.} & The paper is relevant as the thesis will make use of Multi-Objective Genetic Programming and the integration of semantics may be of use in adding constraints for more optimal allocation	The paper focuses mainly on Multi-Objective Genetic Programming and has no objectives in the approaches of shelter allocation nor python programming.\\ \hline
\end{longtable}

\begin{longtable}{|L{.2\linewidth}|L{.74\linewidth}|}
	\hline
	\multicolumn{2}{|L{.94\linewidth}|}{\fullcite{Ghazaleh2022}}\\ \hline
	\textbf{Summary.} & The paper presents a hybrid genetic algorithm combining genetic algorithms (GA) for global search with particle swarm optimization (PSO) for local search in multi-objective optimization problems. It introduces a modified selection mechanism using K-means clustering to enhance the selection of promising solutions and to rehabilitate rejected individuals. The HGA was tested on four benchmark functions, achieving a balance between exploration and exploitation, while also reducing the average iterations needed for convergence, thus improving overall performance.\\ \hline
	\textbf{Critique.} & The paper requires a better and more detailed discussion of computational complexity and real-world applicability.\\ \hline
	\textbf{Similarities and Differences.} & The paper is relevant toward the thesis due to the use of genetic algorithms for search and rescue operations.	The paper focuses on the routing of search and rescue resources after earthquakes, whilst the thesis focuses on the allocation of communities to shelters for protection against natural disasters, not only earthquakes.\\ \hline
\end{longtable}

\begin{longtable}{|L{.2\linewidth}|L{.74\linewidth}|}
	\hline
	\multicolumn{2}{|L{.94\linewidth}|}{\fullcite{Ahmed2020}}\\ \hline
	\textbf{Summary.} & This paper proposes a hybrid genetic algorithm (HGA) in optimizing multi-objective problems. The HGA combines a genetic algorithm with a particle swarm optimization algorithm (PSO) and clustering algorithm. The GA is used to perform global search, while the PSO is used to perform a local search to rehabilitate rejected individuals and the clustering algorithm is used to ensure the fair distribution of individuals in the selection process. The HGA is tested against four benchmark multi-objective optimization functions and compared with a traditional GA.\\ \hline
	\textbf{Critique.} & The paper could benefit from a more detailed explanation of the encoding and decoding process of the problem and solution and the use of clustering algorithm as a secondary selection technique is not fully explained and its impact on the performance of the HGA\\ \hline
	\textbf{Similarities and Differences.} & The proposed HGA can be relevant for its similarities in finding the optimal locations for shelters to minimize the distance between shelter and demand points, considering the factors such as capacity, budget, and accessibility.	The main difference is the incorporation of additional optimization techniques, specifically the PSO and clustering algorithm. The proposed HGA combined the strengths of the genetic algorithm with the local search capabilities of PSO and the clustering capabilities of the clustering algorithm which allows a more efficient and effective search for optimal solutions.\\ \hline
\end{longtable}

\begin{longtable}{|L{.2\linewidth}|L{.74\linewidth}|}
	\hline
	\multicolumn{2}{|L{.94\linewidth}|}{\fullcite{Aitichya2021}}\\ \hline
	\textbf{Summary.} & The article proposes a multi-objective optimization approach to develop Freight Traffic Analysis Zones FTAZs using a genetic algorithm. The decision variables represent freight, passenger traffic, and land use characteristics.\\ \hline
	\textbf{Critique.} & The methodology is good but its application to shelter location allocation may be limited due to the different nature of the problems as the paper focuses on zoning systems which are not directly the same to shelter location allocation.\\ \hline
	\textbf{Similarities and Differences.} & The article is relevant as it applies a similar optimization approach to a different problem domain. The difference is the application domain, with the article focusing on freight traffic analysis zones, whereas shelter location allocation focuses on optimizing shelter locations based on factors such as distance and capacity\\ \hline
\end{longtable}

\begin{longtable}{|L{.2\linewidth}|L{.74\linewidth}|}
	\hline
	\multicolumn{2}{|L{.94\linewidth}|}{\fullcite{Vladislav2022}}\\ \hline
	\textbf{Summary.} & The paper presents a comparison of the performance of three parallelized genetic algorithm models such as Master-Slave, Coarse-Grained and Fine-Grained. The authors implemented the models using the Scalable Concurrent Operation in Python SCOOP model and RabbitMQ for communication.\\ \hline
	\textbf{Critique.} & The paper provides a thorough comparison of the three parallelized GA models. The authors did not provide a detailed analysis of the computational complexity of the models.\\ \hline
	\textbf{Similarities and Differences.} & The paper emphasized the importance of communication between nodes in a parallelized GA model, this is relevant to the shelter location allocation problem that requires the coordination between different locations.	The paper differs on the problem domain.\\ \hline
\end{longtable}
