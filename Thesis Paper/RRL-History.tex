\section{History of Government Response}

This section of the RRL reviews the history of government response toward disasters, these provide insight to strategies that may or may not be effective currently.

\begin{longtable}{|L{.2\linewidth}|L{.74\linewidth}|}
	\hline
	\multicolumn{2}{|L{.94\linewidth}|}{\fullcite{ThesisTag}}\\ \hline
	\textbf{Summary.} & Provide a summary of the reference. Highlight the main findings, methodologies, and conclusion of this work.\\ \hline
	\textbf{Critique.} & Critique the strengths and limitations of the reference. Consider aspects like research design, sample size, data collection methods, and statistical analyses. For the developed system, consider software development methodologies, usability and user experience, security and privacy, and case studies of successful system implementations. Discuss any biases or gaps you identify.\\ \hline
	\textbf{Similarities and Differences.} & Compare and contrast the findings and methodologies of past studies with your own research. What aspects align with your work, and where do you diverge?\\ \hline
\end{longtable}

\begin{longtable}{|L{.2\linewidth}|L{.74\linewidth}|}
	\hline
	\multicolumn{2}{|L{.94\linewidth}|}{\fullcite{WebsiteTag}}\\ \hline
	\textbf{Summary.} & Provide a summary of the reference. Highlight the main findings, methodologies, and conclusion of this work.\\ \hline
	\textbf{Critique.} & Critique the strengths and limitations of the reference. Consider aspects like research design, sample size, data collection methods, and statistical analyses. For the developed system, consider software development methodologies, usability and user experience, security and privacy, and case studies of successful system implementations. Discuss any biases or gaps you identify.\\ \hline
	\textbf{Similarities and Differences.} & Compare and contrast the findings and methodologies of past studies with your own research. What aspects align with your work, and where do you diverge?\\ \hline
\end{longtable}

Include as many literature as needed that falls in the same theme.



