\section{History of Government Response}

This section of the RRL reviews the history of government response toward disasters, these provide insight to strategies in shelter allocation that may or may not be effective currently.

\begin{longtable}{|L{.2\linewidth}|L{.74\linewidth}|}
	\hline
	\multicolumn{2}{|L{.94\linewidth}|}{\fullcite{Flebus1941}}\\ \hline
	\textbf{Summary.} & The article explains the factors and methods that influence the design of shelters in Europe, as well as the levels of shelters and the protection they provide. The article also provides reasoning for the distribution of communities toward what level of shelter.\\ \hline
	\textbf{Critique.} & The article is about shelters that protect toward air raids and bombings, not natural disasters. Also, the reasoning for the distribution of communities toward a specific level of shelter uses either the finances of the individual or the military rank.\\ \hline
	\textbf{Similarities and Differences.} & The proposed literature may be relevant in deciding what a specific shelter in the study is a level 1 or a level 2 shelter, as the article still provides specifications of levels of shelter, albeit at a different system. The main difference is the system of levels of shelters, the article providing 4, as well as the uses of each shelter, providing protection from varying levels of explosive blasts. \\ \hline
\end{longtable}

\begin{longtable}{|L{.2\linewidth}|L{.74\linewidth}|}
	\hline
	\multicolumn{2}{|L{.94\linewidth}|}{\fullcite{Shakibamaesh2015}}\\ \hline
	\textbf{Summary.} & The journal article focuses on the importance of passive defense measures in improving the city's resilience against wartime. Passive defense are non-military actions with the goal of reducing vulnerability from damage, and provide the ability of urban areas to manage crises. The article involves explaining principles of prevention, preparation, and emergency response, as well as a short explanation on the distribution of shelters towards specific areas.\\ \hline
	\textbf{Critique.} & The article provides explanation on shelters that protect from invading forces in wartime. This aspect is not relevant toward the thesis, as wartime and invading forces are not covered.\\ \hline
	\textbf{Similarities and Differences.} & The article could be relevant for its brief explanation on the distribution of shelters in specific areas, this may be used in the thesis to provide insight on where to build or where to assign a shelter. The main difference of the article compared to the thesis is the use of each shelter. The article's shelters are geared towards protection form small arms and gunfire, while the thesis' shelters are for floods and other natural disasters. \\ \hline
\end{longtable}

\begin{longtable}{|L{.2\linewidth}|L{.74\linewidth}|}
	\hline
	\multicolumn{2}{|L{.94\linewidth}|}{\fullcite{Hossain2018}}\\ \hline
	\textbf{Summary.} & The article summarizes the effects of the Bhola cyclone in 1970 toward the political state of Bangladesh. The country was affected tremendously by the cyclone due to its state of economy, with assistance from neighbouring countries either being too few or not effective enough. This has given rise to the independence of Bangladesh from Pakistan. In the years after 1970, steps were taken by the new government to improve disaster readiness, investing significantly in infastructure such as embankments, warning systems, and cyclone shelters. \\ \hline
	\textbf{Critique.} & The paper covered more about the politics related to Bangladesh and Pakistan than the disaster and how shelters were built.\\ \hline
	\textbf{Similarities and Differences.} & This paper may provide insight toward government response to gigantic natural disasters, thus helping the thesis in adjusting based on government response. The paper talks about Bangladesh in the 1970s, which has a different government compared to present day Philippines, researchers must take this into account and adjust as needed.\\ \hline
\end{longtable}

\begin{longtable}{|L{.2\linewidth}|L{.74\linewidth}|}
	\hline
	\multicolumn{2}{|L{.94\linewidth}|}{\fullcite{ThesisTag}}\\ \hline
	\textbf{Summary.} & The article summarizes the response of the government of the Philippines for the vulnerability of the country to natural disasters. The government has developed disaster risk reduction and management (DRRM) strategies, which shifted from reactive to proactive governance. Typhoon Yolanda in 2013 tested the effectiveness of these strategies, revealing shortcomings in coordination and inclusivity. Ongoing discussions focus on strengthening governance to improve disaster preparedness, response, and recovery.\\ \hline
	\textbf{Critique.} & The paper makes no mention nor reference to shelter allocation. Merely the history of government response.\\ \hline
	\textbf{Similarities and Differences.} & The paper being a summarization of the history of government response may provide insight for the researchers in the current state of disaster protection and response laws, as well as provide guidance in what has been attempted before. The paper does not provide disaster allocation tools nor decision support tools for use by the authorities.\\ \hline
\end{longtable}