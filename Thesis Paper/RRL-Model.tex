\section{Mathematical Modeling}

The following articles, papers and journals contain several models that were chosen for use in the thesis. These models may provide insight and advice unto the implementation and application of shelter allocation.


\begin{longtable}{|L{.2\linewidth}|L{.74\linewidth}|}
	\hline
	\multicolumn{2}{|L{.94\linewidth}|}{\fullcite{Hasti2021}}\\ \hline
	\textbf{Summary.} & The paper proposes a two-stage multi-objective mathematical model for integrated disaster preparation and response. The model aims to minimize the total delay time. The NSGA-II and MOVDO are used to solve the model and obtain near-optimal solutions.\\ \hline
	\textbf{Critique.} & "The performance of the proposed model is not compared with other existing models in the literature.\\ \hline
	\textbf{Similarities and Differences.} & The paper is relevant as it proposes a model that can be used to optimize the location of emergency shelters and depots.	The proposed model differs in the consideration of multiple objectives and integration of disaster preparation and response. The model also considers the allocation of shelters to depots and the distribution of relief supplies.\\ \hline
\end{longtable}

\begin{longtable}{|L{.2\linewidth}|L{.74\linewidth}|}
	\hline
	\multicolumn{2}{|L{.94\linewidth}|}{\fullcite{Panchalee2021}}\\ \hline
	\textbf{Summary.} & The paper proposes a multi-objective optimization model for shelter location-allocation in response to humanitarian relief logistics. The model aims to minimize total cost, evacuation time, and the number of open shelters. The Epsilon constraint EC method and Goal programming GP are used to solve the model. A case study about the flooding in Surat Thani in Thailand is used to validate the applicability of the proposed model.\\ \hline
	\textbf{Critique.} & The paper assumed that the number of victims in each affected area is known and fixed, which may not be the case in real-world scenarios. Additionally, the model does not consider the capacity of  the shelters and is limited to a single scenario.\\ \hline
	\textbf{Similarities and Differences.} & The paper is relevant in the sense that both approaches the aim to optimize shelter location-allocation. 	The proposed shelter location allocation is using mathematical programming approach with EC and GP. \\ \hline
\end{longtable}

\begin{longtable}{|L{.2\linewidth}|L{.74\linewidth}|}
	\hline
	\multicolumn{2}{|L{.94\linewidth}|}{\fullcite{Jian2021}}\\ \hline
	\textbf{Summary.} & The article discusses utilizing existing underground spaces as emergency shelters in urban community areas, focusing on optimizing shelter locations and pedestrian evacuation routes. A network flow model is developed to maximize evacuation demand, with a real-world case study in Shanghai, China demonstrating the effectiveness of the proposed methodology.\\ \hline
	\textbf{Critique.} & The study assumed the fixed pedestrian speed and does not consider other factors that may affect evacuation.\\ \hline
	\textbf{Similarities and Differences.} & The relevance is the optimizing shelter locations and evacuation routes in emergency situations but the article uses a network flow model.	The use of a networking flow model and modified algorithm and focuses on utilizing underground spaces and sidewalk capacity.\\ \hline
\end{longtable}

\begin{longtable}{|L{.2\linewidth}|L{.74\linewidth}|}
	\hline
	\multicolumn{2}{|L{.94\linewidth}|}{\fullcite{Yiying2022}}\\ \hline
	\textbf{Summary.} & The paper proposes a multi-objective location decision-making model for emergency shelters that prioritizes the subjective evaluation of residents. The subjective scores of residents for each attribute are obtained using a combination of normal distribution, ordered weighted aggregation OWA operator, and analytic hierarchy process AHP.\\ \hline
	\textbf{Critique.} & The paper assumed the subjective scores of residents for each attribute as independent. The model does not consider the capacity constraint of emergency shelters which is an important factor in shelter location allocation.\\ \hline
	\textbf{Similarities and Differences.} & Compare and contrast the findings and methodologies of past studies with your own research. What aspects align with your work, and where do you diverge?"The paper is relevant to shelter location allocation as it proposes a multi-objective model that can be solved using genetic algorithms. The paper differs in the use of the combination of normal distribution, OWA operator, and AHP to obtain the subjective scores of residents, whereas genetic algorithm-based models typically use fitness function to evaluate the quality of solutions. Additionally, the paper does not use genetic algorithms to solve the model.\\ \hline
\end{longtable}

\begin{longtable}{|L{.2\linewidth}|L{.74\linewidth}|}
	\hline
	\multicolumn{2}{|L{.94\linewidth}|}{\fullcite{Maria2023}}\\ \hline
	\textbf{Summary.} & The study focuses non formulating a Complex Non-linear Diophantine Fuzzy (CN-LDF) decision-making model  and apply it on earthquake shelter construction. They discuessed the preliminaries for understanding these concepts and compare the existing models to their proposed models. The approach was applied in a case study on emergency earthquake shelter material selection, where plastic sheeting emerged as the best option.\\ \hline
	\textbf{Critique.} & The concepts introduced were hard but this may targeted on advanced researchers. They also introduced so many model variations of CN-LDF considering that they claimed to be the first one to introduce such method in decisio-making model.\\ \hline
	\textbf{Similarities and Differences.} & This could enhance the integrity of picking the right shelter for evacuation. It also have the same objectives of forumlating a model then testing in on a natural hazard case. The paper focuses on materials to be used in shelters and has completely different models used. It is also not a system, but a numerical analysis research.\\ \hline
\end{longtable}

\begin{longtable}{|L{.2\linewidth}|L{.74\linewidth}|}
	\hline
	\multicolumn{2}{|L{.94\linewidth}|}{\fullcite{Xiujuan2019}}\\ \hline
	\textbf{Summary.} & The study formulated a hierarchical mathematical model of  earthquake shelter location-allocation problem. It features emergency shelter and long-term shelter that could transfer evacuees if needed. The model is solved by interleaved modified particle swarm optimization algorithm and genetic algorithm (MPSO–GA) and applied in Beijing, China.\\ \hline
	\textbf{Critique.} & The model greatly identified which shelter are emergency purposes or for long lasting int terms of distance, cost, and also the impact of the earthquake. They did not however consider the construction and neighboring facilities like if the area has an hospital or relief operations.\\ \hline
	\textbf{Similarities and Differences.} & Our model ere derived from their model, by determining level 1 or level 2 shelter.  We could simplify the model to integrate it to our system.	The study used MPSO as initial population, but on ours we will only use a much simpler approach. The paper also doesn't developed a system.\\ \hline
\end{longtable}

\begin{longtable}{|L{.2\linewidth}|L{.74\linewidth}|}
	\multicolumn{2}{|L{.94\linewidth}|}{\fullcite{Yunjia2019}}\\ \hline
	\textbf{Summary.} & The paper exposes different models (Single-Objective, Multi-objective, \& Hierarchical) to best represent a shelter location-allocation problem. The following models are compared on which objectives they minimized and maximized, and what algorithm are used to solved them. No algorithm seems to be the best at solving the problem since each algorithm has their own strengths and weaknesses.\\ \hline
	\textbf{Critique.} & The study listed all of their scoped models and even compare who are the authors for each models. This is a great starting point for picking which model and algorithm is suitable for a problem\\ \hline
	\textbf{Similarities and Differences.} & The simplified models are listed on this paper. The model we are adopting are derived from the study. This could help justify our methodologies.	The paper covered much simpler models. The models we will use are combination of those models.\\ \hline
\end{longtable}

\begin{longtable}{|L{.2\linewidth}|L{.74\linewidth}|}
	\hline
	\multicolumn{2}{|L{.94\linewidth}|}{\fullcite{UnknownAuthor}}\\ \hline
	\textbf{Summary.} & The study formulated 4 shelter location-allocation models derived from GEN and HIER models. BNST, BST, BNT, \& WORK model are solved using binary genetic algorithm and applied to the province of Talisay, Batangas. Numerical analaysis were made to visualize the performance of the models.\\ \hline
	\textbf{Critique.} & They only used simulated data for shelter sites. This may alter the performance results of the model if real data was applied.\\ \hline
	\textbf{Similarities and Differences.} & This is the paper we are basing our study on. We will use the BNT model and gather real data from our target location.	The paper used binary genetic algorithm, but we chose to use index parameters since it is dynamic and much faster. The paper also doesn't developed a system.\\ \hline
\end{longtable}

\begin{longtable}{|L{.2\linewidth}|L{.74\linewidth}|}
	\hline
	\multicolumn{2}{|L{.94\linewidth}|}{\fullcite{Yunjia2020}}\\ \hline
	\textbf{Summary.} & The study formulated a shelter location-allocation model but taking consider of typhoon track. The proposed model is composed of several static processes and solved by MPSO with a restart strategy. The model applied in Typhoon Rammasun and Mirinae on the province of Wenchang,Hainan of China.\\ \hline
	\textbf{Critique.} & They used MPSO with restart strategy to avoid stucking in a local optima. They could used genetic algorithm with decaying mutation rate instead.\\ \hline
	\textbf{Similarities and Differences.} & The objectives of the model were similar to our model. The way they exposed the model might help us in our model methodology.	The model took consider of typhoon track and weather forecasting. They also used different algorithm to solve the model. They also didn't mention any system developed.\\ \hline
\end{longtable}

\begin{longtable}{|L{.2\linewidth}|L{.74\linewidth}|}
	\hline
	\multicolumn{2}{|L{.94\linewidth}|}{\fullcite{Peiman2019}}\\ \hline
	\textbf{Summary.} & The study proposed an uncertain multi-objective multi-commodity multi-period multi-vehicle location-allocation model and applied in a real case study in Tehran, Iran.  They solved the model using MMOPSO, NSGA-II, and Epsilon constraint method. This model offers a comprehensive approach to disaster response logistics, balancing cost efficiency with the need to minimize shortages of supplies.\\ \hline
	\textbf{Critique.} & They might consider routing of relief distribution and evacuation of victims in the problem can be an interesting future direction.\\ \hline
	\textbf{Similarities and Differences.} & The objectives of the model were similar to our model. The findings that MMOPSO algorithms outperforms any algorithms might help us took consider and justify the sai algorithm.	They took many variables and objectives such as vehicle, period, and injuries. They also applied three algorithms. They did not however developed a system.\\ \hline
\end{longtable}

\begin{longtable}{|L{.2\linewidth}|L{.74\linewidth}|}
	\hline
	\multicolumn{2}{|L{.94\linewidth}|}{\fullcite{Reza2020}}\\ \hline
	\textbf{Summary.} & The study investigates the distribution and redistribution of relief goods in post-disaster scenarios, focusing on Vehicle Routing Problem (VRP) and Network Flow Problem (NFP) structures while emphasizing fairness in distribution. A specialized simulated annealing algorithm, enhanced with crossover operators, is proposed. Additionally, a systematic three-phase technique is employed for tuning the algorithm. The model is validated through testing on sample problems and a case study.\\ \hline
	\textbf{Critique.} & The study addresses the distribution and redistribution of relief goods, however victims are not taken into account.\\ \hline
	\textbf{Similarities and Differences.} & The paper may be of use due to VRP and NFP being a possible model for the thesis.	The main difference of both papers are the goals, with the study about the distribution of relief goods, and the thesis being about the distribution of victims toward appropriate shelters.\\ \hline
\end{longtable}
