\section{Shelter Location-Allocation Framework}

The following studies and articles present various real-world scenarios that demonstrate the application of identifying shelters during disasters and allocating affected communities to said shelters. 

\begin{longtable}{|L{.2\linewidth}|L{.74\linewidth}|}
	\hline
	\multicolumn{2}{|L{.94\linewidth}|}{\fullcite{Shaoqing2020}}\\ \hline
	\textbf{Summary.} & The study focuses on optimizing emergency shelter allocations and resource allocation in the aftermath of natural disasters. The integration of qualitative factors, fuzzy AHP, and fuzzy TOPSIS methods which enhance decision making by expert judgments.\\ \hline
	\textbf{Critique.} & While the study effectively addresses the needs of disaster victims within limited resources, it primarily focuses on single-cycle decisions.\\ \hline
	\textbf{Similarities and Differences.} & The paper is another shelter allocation model that may be of use to the researchers	The main difference between the paper and the thesis are the models which are used. The paper uses fuzzy AHP and fuzzy TOPSIS whilst the thesis will make use of Genetic Algorithms and HIER models \\ \hline
\end{longtable}

\begin{longtable}{|L{.2\linewidth}|L{.74\linewidth}|}
	\hline
	\multicolumn{2}{|L{.94\linewidth}|}{\fullcite{Panchalee2021}}\\ \hline
	\textbf{Summary.} & The article proposes a novel approach for determining shelter location-allocation in humanitarian relief logistics using a combination of the epsilon constraint method EC and artificial neural networks ANN. The approach aims to minimize total cost and total evacuation time simultaneously, where EC is used  to solve a multi-objective optimization model and ANN will learn from the optimal solutions generated by the EC. This is demonstrated through a case study of shelter location-allocation in response to flooding in five districts of Surat Thani, Thailand.\\ \hline
	\textbf{Critique.} & The proposed methodology relies heavily on the quality of the data used to train the ANN. Additionally, the case study is limited to a specific region and may not be generalizable to other areas.\\ \hline
	\textbf{Similarities and Differences.} & The article focuses on shelter location-allocation which is relevant to the thesis, as both methods aim to optimize shelter location-allocation.	The methodology differ in the use of a combination of EC and ANN. \\ \hline
\end{longtable}

\begin{longtable}{|L{.2\linewidth}|L{.74\linewidth}|}
	\hline
	\multicolumn{2}{|L{.94\linewidth}|}{\fullcite{Lei2022}}\\ \hline
	\textbf{Summary.} & The paper proposes a bi-level multi-objective location-allocation model, called AEE model to optimize the location of urban shelter. The model considers three core objectives: fairness, efficiency, and cost. The model uses a gravity model to simulate the decision making behavior of evacuees and consider the shelters based on their distance and scale. The model is solved using a simulated annealing algorithm SAA and is applied to a case study in China.\\ \hline
	\textbf{Critique.} & The paper assumes the decisions of the evacuees based on the distance and scale of shelters. The model does not consider other important factors that may affect shelter.\\ \hline
	\textbf{Similarities and Differences.} & The AEE model is relevant to shelter location allocation as both aim to optimize the location of shelters to minimize evacuation time and cost.	The AEE model differs as it uses a bi-level programming approach; the model also uses a gravity model to simulate the decision-making behavior of evacuees whereas genetic algorithms do not. Additionally, the AEE model is solved using SAA, whereas the genetic algorithm uses a population based search approach. \\ \hline
\end{longtable}

\begin{longtable}{|L{.2\linewidth}|L{.74\linewidth}|}
	\hline
	\multicolumn{2}{|L{.94\linewidth}|}{\fullcite{Yin2023}}\\ \hline
	\textbf{Summary.} & The study is about formulating shelter allocation model, and solving it by using Improved Quantum Genetic Algorithm (IQGA). They also integrated a spreading operator used in the traditional allocation method, spreading model. This assess their model on which has 3 objective functions, distance, discrete distance, and the rationality. The allocation model with the spreading operation produces better results than the one without. Additionally, evacuation route and heat maps are generated to visualize the shelter distribution.\\ \hline
	\textbf{Critique.} & The formulated model lacks reference models, which raises questions about the credibility of the objectives used.\\ \hline
	\textbf{Similarities and Differences.} & The study has the same objectives as our study. This could be helpful in enhancing our study's methodologies. It also uses a more advanced genetic algorithm, IQGA, thus in the future, we might integrate the algorithm for faster and more efficient solution generation. 	The study has entirely different objective functions for their mathematical model. The model formulated were solved by IQGA, and compared with spreading model. It also doesn't have a system. \\ \hline
\end{longtable}

\begin{longtable}{|L{.2\linewidth}|L{.74\linewidth}|}
	\hline
	\multicolumn{2}{|L{.94\linewidth}|}{\fullcite{Kuo2024}}\\ \hline
	\textbf{Summary.} & The study formulated a two-stage stochastic model, on which the first stage determining the location of the shelters, allocation of the evacuees as the second stage. It aims to minimize costs with considering different factors such as their evacuee behavior, population needs, and traffic. The framework provides valuable observations and insights into optimal shelter location and allocation under post-disaster conditions. \\ \hline
	\textbf{Critique.} & The researchers chose to use a simulated data to test their model. They generated it by randomness. It may be better if they have a real data. \\ \hline
	\textbf{Similarities and Differences.} & This study has more objectives to consider in finding the optimal shelter allocation. This may help to improve the decision making output of our model.	The paper highlights more objectives such as evacuee behavior and post-disaster scenario. It also used simulated data. and has different approach of solving it by using OCBA.  \\ \hline
\end{longtable}

\begin{longtable}{|L{.2\linewidth}|L{.74\linewidth}|}
	\hline
	\multicolumn{2}{|L{.94\linewidth}|}{\fullcite{Elita2022}}\\ \hline
	\textbf{Summary.} & The study aims to formulate a model that minimize operational costs considering allocation of health workers and injured victims after an earthquake. The model were solve using LINGO 18.0 software and applied to a real case study in Padanag, Indonesia. The results show that reducing the number of victims at evacuation posts lowers transportation costs for victims and health workers, while increasing health facility capacity raises operational costs. \\ \hline
	\textbf{Critique.} & They might consider adding a green triage which represent the minor injured victims \\ \hline
	\textbf{Similarities and Differences.} & The allocation model on which summation of objectives were similar and relevant to our model.	They added considerations in health care workers, and used LINGO 18.0 to solve the model.  \\ \hline
\end{longtable}

\begin{longtable}{|L{.2\linewidth}|L{.74\linewidth}|}
	\hline
	\multicolumn{2}{|L{.94\linewidth}|}{\fullcite{Nicolas2024}}\\ \hline
	\textbf{Summary.} & The study proposes a risk-based multidimensional model to help decision-makers select the best shelter locations in terms of flood disaster. The methodology incorporates  Decision Analysis and MAUT to assess the potential shelter sites. Their model has the potential to adapt to any urban context and encourages decision-makers to integrate visualization tools and sensitivity analysis into their model. \\ \hline
	\textbf{Critique.} & They validated the proposed model by applying it to a case study in Recife,Brazil.,but they did not assess the model by directly asking the decision makers regarding its output. They only rely to the numerical analysis of the model. \\ \hline
	\textbf{Similarities and Differences.} & This considers the distance, evacuation flow during the evacuation phase in shelter allocation model. It may help in enhancing our methodologies.	The paper differs in methodologies used such as Decision Analysis and MAUT. The paper also doesn't developed a system. \\ \hline
\end{longtable}

\begin{longtable}{|L{.2\linewidth}|L{.74\linewidth}|}
	\hline
	\multicolumn{2}{|L{.94\linewidth}|}{\fullcite{Alexandra2021}}\\ \hline
	\textbf{Summary.} & "The study aims to compare the relative suitability of different school buildings for these purposes by using AHP based on the combined opinions of relevant experts. Among 38 school buildings in Cagayan de Oro, issues include poor nighttime evacuation access, mobility barriers, and limited alternative spaces, disrupting education.\\ \hline
	\textbf{Critique.} &They may extend the research to asking the evacuees or residents instead. Then, they can compare of which weights are actually better and suitable. \\ \hline
	\textbf{Similarities and Differences.} & The criterion used on selecting shelters might help us to determine the candidate shelter sites. This could help the us in data collection if the shelter is not yet built.	The study only focused on the characteristics and suitability of a candidate shelter, specifically school. \\ \hline
\end{longtable}
