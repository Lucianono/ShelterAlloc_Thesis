\section{Decision Support Systems}

The systems used in the following articles and papers have been selected since it features shelter location-allocation models and solution.  These may be applied or provide insight on designing and developing our proposed programmed system.

\begin{longtable}{|L{.2\linewidth}|L{.74\linewidth}|}
	\hline
	\multicolumn{2}{|L{.94\linewidth}|}{\fullcite{Kim2019}}\\ \hline
	\textbf{Summary.} & The article critiques the use of Distributed Evolutionary Algorithm in Python.  DEAP supports Genetic Algorithm and other formulas. An easy to use GP library that is written in Python and can be used for data processing and machine learning.\\ \hline
	\textbf{Critique.} & The article is extremely short and only highlights the features, strengths and weaknesses of the Distributed Evolutionary Algorithm in Python. The article does not teach the reader how to use or apply the algorithm in a meaningful way.\\ \hline
	\textbf{Similarities and Differences.} & The Distributed Evolutionary Algorithm in Python library supports evolutionary algorithms including Genetic Algorithm. However, the article makes no mention of the library being used for location allocation nor for optimal allocation. \\ \hline
\end{longtable}

\begin{longtable}{|L{.2\linewidth}|L{.74\linewidth}|}
	\hline
	\multicolumn{2}{|L{.94\linewidth}|}{\fullcite{Cavdur2019}}\\ \hline
	\textbf{Summary.} & This paper proposes the use of a decision support tool for the allocation of temporary disaster-response facilities while under the effects of demand uncertainty.  The paper also develops a database for storage of disaster and shelter details and for use by disaster operations.\\ \hline
	\textbf{Critique.} & The paper makes no reference as to the language used in programming the backend of the system. Only the software used for the interface, the core system, and the database.\\ \hline
	\textbf{Similarities and Differences.} & The paper is similar to the thesis, being a shelter allocation related decision support tool created for the use of disaster-response members. The paper makes no mention of Genetic Algorithm and of the programming language used. Also, the paper allocates TDR facilities for relief supplies distribution.\\ \hline
\end{longtable}

\begin{longtable}{|L{.2\linewidth}|L{.74\linewidth}|}
	\hline
	\multicolumn{2}{|L{.94\linewidth}|}{\fullcite{Fatih2020}}\\ \hline
	\textbf{Summary.} & The study creates a spreadsheet based decision support tool for the allocation of temporary disaster response facilities for the distribution of relief supplies. Developed in a spreadsheet environment for easier modification of data.\\ \hline
	\textbf{Critique.} & The use of a spreadsheet instead of an application creates a higher learning curve that will be harder for different authorities to read and write.\\ \hline
	\textbf{Similarities and Differences.} & The study is relevant in the sense that it is also a decision support system for allocation. The study contains allocation for the distribution of relief supplies instead of victims, as well as the use of a spreadsheet instead of a program. \\ \hline
\end{longtable}

\begin{longtable}{|L{.2\linewidth}|L{.74\linewidth}|}
	\hline
	\multicolumn{2}{|L{.94\linewidth}|}{\fullcite{Kurt2021}}\\ \hline
	\textbf{Summary.} & The paper proposes the optimal location of COVID-19 vaccination sites at the municipal level in the Philippines by using a dynamic optimization approach to recalculate the optimal vaccination sites as the population of barangays that have completed their vaccination program changes. The study uses a single-objective optimization problem and a bi-objective optimization problem to minimize the distance to vaccination sites and maximize the coverage of the population.\\ \hline
	\textbf{Critique.} & The study assumed the static population with the lack of consideration of other costs associated with vaccine delivery.\\ \hline
	\textbf{Similarities and Differences.} & The paper approached the optimization of vaccination site locations by the relevance to shelter location allocation problems, as both involve allocation resources to minimize cost and maximize benefits. The study differs with its objective function, problem constraints, algorithm implementation, and context. The study focused on maximizing population coverage in the context of COVID-19 vaccination while shelter location allocation problems aim to maximize the capacity of shelters in disaster scenarios. \\ \hline
\end{longtable}

\begin{longtable}{|L{.2\linewidth}|L{.74\linewidth}|}
	\hline
	\multicolumn{2}{|L{.94\linewidth}|}{\fullcite{Tuğçe2022}}\\ \hline
	\textbf{Summary.} & The paper proposes a decision support tool for public emergency scenarios to ensure accessibility to vital resources. The decision support tool generates possible locations to allocate missing requested services on the intersection points of available requested services by considering the population density.\\ \hline
	\textbf{Critique.} & The paper’s scope is limited to allocating temporary facilities and made the assumptions considering population density as the sole factor affecting allocation. Furthermore, the paper lacks validation and testing of the proposed tool.\\ \hline
	\textbf{Similarities and Differences.} & The paper focuses on allocating temporary facilities in emergency scenarios relevant to shelter location allocation.The proposed decision support tool differs from shelter location allocation that uses a database-driven approach, while the paper focuses solely on allocating temporary facilities. \\ \hline
\end{longtable}

\begin{longtable}{|L{.2\linewidth}|L{.74\linewidth}|}
	\hline
	\multicolumn{2}{|L{.94\linewidth}|}{\fullcite{Sara2022}}\\ \hline
	\textbf{Summary.} & The research introduced a DSS called EDIS to helps estimate the needs of affected populations, because of continuing problems on uncoordinated influx of suppliers. The system uses historical data from previous disaster responses in four humanitarian sectors WASH, Nutrition, Health, and Shelter, on which methodologies such as MADM, AHP, and MAUT are used. The system outperforms existing methods by providing quick, reliable, and adaptable estimations using public and historical data, thus speeding up disaster responses by 72 hours.\\ \hline
	\textbf{Critique.} & The paper has very good impact on community and directly assessed the problem they're seeking. However, the system were not shown in the paper and no system structure is introduced. It is due to they only developed a framework, but not a fully functional system.\\ \hline
	\textbf{Similarities and Differences.} & The targeted beneficiaries were same, and the development of a decision support system/ framework  is the same. The data they needed are considered by balancing supply and demand of resources. This might be helpful for our data collection methods. The paper is focused on humanitarian logistics, and has completely different methodologies. The paper also focused on the data produced of the system. \\ \hline
\end{longtable}

\begin{longtable}{|L{.2\linewidth}|L{.74\linewidth}|}
	\hline
	\multicolumn{2}{|L{.94\linewidth}|}{\fullcite{Amir2023}}\\ \hline
	\textbf{Summary.} & The study introduced risk-based decision support system to help disaster risk management planners select optimal locations for emergency shelters after an earthquake. Considering 18 criterion, LGDM model were used to gathers input from a large group of decision-makers or experts to assign importance levels to each criterion. Next, OWA method were used to determine the suitability map. The system has GIS integration and risk attitude from decision makers.\\ \hline
	\textbf{Critique.} & The paper assess almost all criterion needed for shelter identification, which is good for finding the most suitable shelter for the residents. The paper however didn't show the system structure, and the system itself. The system also has no acceptability measures.\\ \hline
	\textbf{Similarities and Differences.} & Beneficiaries are same. Shelter identification is crucial in our study. We might consider their criterion to identify the characteristics of shelter we may need on our study. The paper has different methodology by using LGDM and it is not a single objective model. The paper did not focused on the system's performance and structure. \\ \hline
\end{longtable}
