\section{Theoretical Framework}

The development of the shelter location-allocation system draws on key theories which feature the optimization of shelter location-allocation, development, and assessment of the system.

	\textbf{Operations Research} Mathematically modeling is an approach to represent a real-life problem and solving them using mathematical techniques such as the operations research. This theory addresses the optimization of objectives, crucial in shelter location-allocation where factors like distance and cost must be balanced. The problem is modeled to find the optimal locations while minimizing travel distances and associated costs, making it ideal for a multi-objective optimization.
	
	\textbf{Theory of Evolution} Inspired by Charles Darwin’s theory of evolution, evolutionary algorithms such as genetic algorithms simulate natural selection within code to solve complex optimization problems effectively. This approach is widely used due to its robustness and adaptability, and it will be applied to solve the multi-objective model for shelter location-allocation.
	
	\textbf{Decision Theory} Decision theory is a study of having a rational choice by using models and tools.. This involves creating a system that serves as a decision support tool for decision-makers, such as the PDRRMC, in the allocation of shelters. It features user-defined parameters for model customization and provides functionality for creating, reading, updating, and deleting data.
	
	\textbf{Software Product Assessment Framework (ISO 25010)} ISO 25010 provides standards for assessing software quality, covering aspects such as functionality, usability, reliability, and efficiency. This framework will guide the evaluation of our decision support system through surveys, ensuring it meets high-quality standards.
	 
